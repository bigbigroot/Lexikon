\normalfont\Huge\sffamily\center {Zu diesem Buch} \par

\thispagestyle{empty}
 \vspace*{3cm}
 \normalfont\normalsize
 \raggedright
 \indent
 中国和德国都是世界上具有重大影响力的国家,随着传播 通讯技术的改进,交通技术的进步和经济的高度全球化,两国的合作越来越频繁。 然而,由于文化背景的不同,两国人民在交 往的过程中不能够相互理解,,导致交际不能顺利进行。针对中德工程师学院和德国的应用技术大学之间日益频繁的交流活动,本项目成员作为中德工程师学院的学生希望将一些所闻所知的一些信息整理、归类,尽可能为两国的学生做一些提示。本书不求包罗万象,但求准确,详尽,有趣。

\par
 在上学期,我们小组成员在项目发起人Frau Schneider的引领和指导下,为了增进我院及合作院校中、德两国学生间的文化认同感,减轻乃至消除文化差异所带来的沟通障碍,同时也为了我院新生能够更快适应和融入国际化的教学方式,多元化的人文环境,开展了一系列关于中德文化差异的研讨会、采访、问卷调查和现场调研,最终决定将项目的视角聚焦在七个文化主题上:服饰、饮食、体育、工作、交通、文化、卫生,以web为载体,以百科全书式的写作手法,从学生的视角出发,独特地展现出我们对于异国文化的认识与思考。在项目期间,我们充分运用所学项目管理的知识,通过Projekt Auftrag, Zeitplan, PSP等项目管理工具控制项目进程;通过查阅相关资料、与德国师生互相交流讨论,并根据实际的生活体验,确定了各项主题的具体内容,并定期向同学与老师展示工作成果。项目的最后,我们的成果是喜人的,我们的中德文化差异百科全书在学院官网上线,向中德师生展现了中德文化大花园的一隅,虽然项目仍有不少改进的空间,但是我们希望通过这个学期的共同努力,把中德文化差异百科这座文化桥梁打造的更加坚固、优美。
 \par
 中国,是东方的一条巨龙,历史久远,底蕴深厚;德国,是欧洲一颗闪耀的明珠,后起之秀。在不断发展的道路上,两国不期而遇,相知相识。于是便有了两国文化的交融,两国人民都带着好奇的眼光去了解,去深入,虽然中德文化存在较多差异,但是两国都秉持着相互理解,相互包容,相互学习的态度,文化的相识,使得两国就像亲兄弟一般,紧紧地联系在一起。信任,才会有合作,有合作,才能双赢。信任,基于了解,恰恰文化上的认识,加深了双方的了解,有了基础,使得两国不论是在科技层面,经济层面还是在政治层面都能更加信任,合作双赢也是水到渠成。从大格局来看,文化上的交流,互惠两国,减少两国知识和见识上的差异阻碍,本质上推动两国发展。所以,通过本项目的学习,中德双方能进一步了解对方,求同存异,互惠互利。
\vspace{\baselineskip}
\begin{flushright}\noindent
HangZhou, Mai.  2019\hfill {\texttt Alle Mitgliede des Projekt4}\\
\end{flushright}

\normalfont\Huge\sffamily\center {Zu diesem Buch} \par

\thispagestyle{empty}
 \vspace*{3cm}
 \normalfont\normalsize
 \justify
Sowohl China als auch Deutschland sind L\"ander mit einem bedeutenden Einfluss auf der Welt. Mit der Verbesserung der Kommunikationstechnologie, der Weiterentwicklung der Transporttechnologie und der starken Globalisierung der Wirtschaft ist die Zusammenarbeit zwischen den beiden Ländern immer häufiger geworden. Aufgrund des unterschiedlichen kulturellen Hintergrunds können sich die Menschen der beiden Länder jedoch nicht während des Geschlechtsverkehrs verstehen und die Kommunikation kann nicht reibungslos durchgeführt werden. Angesichts der immer häufiger werdenden Austauschaktivitäten zwischen CDAI und der Deutschen Fachhochschule hoffen wir, einige der Informationen, die wir für die Studenten beider L\"ander so gut wie möglich kennen und ordnen, zu ordnen und zu klassifizieren. Einige Tipps. Dieses Buch soll nicht allumfassend sein, aber es ist genau, detailliert und interessant. Diese Broschüre soll nicht allumfassend sein, aber es ist genau, detailliert und interessant.
\par
Das Chinesisch-Deutschen Institut für Angewandte Ingenieurwissenschaften (CDAI) wurde Anfang Mai im Rahmen eines Treffens von Vertretern der Fachhochschulen Lübeck und Westküste und der Zhejiang University for Science and Technology (ZUST) gestartet. Das CDAI ist in die langjährige erfolgreiche Partnerschaft des Landes Schleswig-Holsteins mit der Provinz Zhejiang eingebettet. Im Fokus des Instituts stehen anwendungsorientierten Bachelor-Ingenieurstudiengänge der schleswig-holsteinischen Fachhochschulen: – Management und Technik der FH Westküste in Heide sowie – Bauingenieurwesen der FH Lübeck. Ca. ein Drittel der fachlichen Lehrveranstaltungen in Hangzhou werden von der jeweiligen Mutterhochschule unterstützt. Die Unterrichtssprache ist zunächst Chinesisch, ab dem dritten Semester kommt Deutsch hinzu und ab dem dritten Studienjahr wird komplett auf Deutsch unterrichtet. Den erfolgreichen Teilnehmern der Studiengänge winkt ein deutscher und ein chinesischer Bachelorabschluss. Für die besten Studierenden besteht die Möglichkeit, im zweiten Studienabschnitt für drei Semester an die jeweilige deutsche Fachhochschule zu wechseln, um das Studium dort zu beenden. Im Gegenzug sollen deutsche Studierende aus dem entsprechenden Studiengang der Partnerhochschule einen Teil ihres Studiums in Hangzhou absolvieren.

\par
Im letzten Semester förderten wir unter Leitung von Projektsponsorin Frau Schneider den kulturelle Verstand zwischen chinesischen und deutschen Studenten in CDAI und deutsche partnerschaftliche Fachhochschule, die durch kulturelle Unterschiede verursacht wurden. Gleichzeitig wurden eine Reihe von Seminaren, Interviews, Fragebögen und Untersuchungen vor Ort zu den kulturellen Unterschieden durchgeführt, um sich an den internationalen Unterrichtsstil und das diversifizierte humanistische Umfeld anzupassen und zu integrieren.  Endlich konzentriert die Perspektive sich auf sieben kulturelle Themen: Kleidung, Ernährung, Sport, Arbeit, Transport, Kultur und Gesundheit und Schreiben wir eine Wikibasiertes Webseite.Aus Sicht der Studenten zeigen wir einmalig unser Verständnis und Denken über fremde Kulturen. Während des Projekts haben wir das Wissen des Projektmanagements voll genutzt und den Projektprozess durch Projektmanagement-Tools wie Projektauftrag, Zeitplan, PSP gesteuert, durch Absprache mit relevanten Materialien, Austausch mit deutschen Lehrern und Schülern und nach der tatsächlichen Lebenserfahrung bestimmt. Die spezifischen Inhalte jedes Themas und zeigen regelmäßig die Arbeitsergebnisse für Klassenkameraden und Lehrer. Am Ende des Projekts sind unsere Ergebnisse erfreulich: Unsere chinesisch-deutsche Enzyklopädie für kulturelle Unterschiede befindet sich auf der offiziellen Website des Kollegs und zeigt den chinesischen und deutschen Lehrern und Schülern einen Einblick in den chinesisch-deutschen Kulturgarten. Wir hoffen, dass durch die gemeinsamen Anstrengungen dieses Semesters die kulturelle Brücke zwischen China und Deutschland stärker und schöner wird.Verglichen mit dem letzten Semester beginnt dieses Semester mit den Unterschieden im täglichen Verhalten der Chinesen und Deutschen und analysiert die kulturellen Konnotationen der einzelnen Unterschiede. Wir hoffen, dass chinesische und deutsche Studenten durch diese Broschüre ihr Verständnis verbessern werden.

\par
China ist ein Long im Osten, hat eine lange Geschichte und eine tiefe Geschichte und Deutschland ist eine glänzende Perle in Europa und ein aufgehender Stern. Auf dem Weg der kontinuierlichen Entwicklung trafen sich die beiden Länder unerwartet und kannten sich. Die beiden Länder haben also eine Mischung von Kulturen: Die beiden Völker haben ein neugieriges Auge, um zu verstehen und tiefer zu gehen: Obwohl es zwischen der chinesischen und der deutschen Kultur viele Unterschiede gibt, haben beide Länder gegenseitiges Verständnis, gegenseitige Toleranz und gegenseitige Lerneinstellung. Die Bekanntschaft der Kultur macht die beiden Länder als Bruder eng miteinander verbunden. Vertrauen, es wird Zusammenarbeit, Zusammenarbeit und Win-Win geben. Vertrauen, basierend auf Verständnis, nur ein kulturelles Verständnis, vertieft das Verständnis beider Seiten und hat eine Grundlage, die die beiden Länder auf wissenschaftlicher, wirtschaftlicher und politischer Ebene vertrauenswürdiger macht. Kooperation und Win-Win sind ebenfalls selbstverständlich. Aus der Perspektive des großen Ganzen haben der kulturelle Austausch, die wechselseitigen Länder und die Verringerung der Unterschiede bei Wissen und Wissen zwischen den beiden Ländern die Entwicklung der beiden Länder wesentlich gefördert. Daher können sich China und Deutschland durch die Studie dieses Projekts besser verstehen, nach Gemeinsamkeiten suchen, dabei Unterschiede aufheben und gegenseitigen Nutzen erzielen.
\vspace{\baselineskip}
\begin{flushright}\noindent
HangZhou, Mai.  2019\hfill {\texttt Alle Mitgliede des Projekt4}\\
\end{flushright}

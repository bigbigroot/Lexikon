\normalfont\Huge\sffamily\center {Zu diesem Buch} \par

\thispagestyle{empty}
 \vspace*{3cm}
 \normalfont\normalsize
 \justify
Sowohl China als auch Deutschland sind Länder mit einem bedeutenden Einfluss auf der Welt: Mit der Verbesserung der Kommunikationstechnologie, der Weiterentwicklung der Transporttechnologie und der starken Globalisierung der Wirtschaft ist die Zusammenarbeit zwischen den beiden Ländern immer häufiger geworden. Aufgrund des unterschiedlichen kulturellen Hintergrunds können sich die Menschen der beiden Länder jedoch nicht während des Geschlechtsverkehrs verstehen und die Kommunikation kann nicht reibungslos durchgeführt werden. Als Antwort auf den immer häufiger werdenden Austausch zwischen dem Deutsch-Chinesischen Institut für Ingenieure und der Deutschen Hochschule für Angewandte Wissenschaften hoffen die Projektmitglieder als Studenten des Deutsch-Chinesischen Instituts für Ingenieure, einige der Informationen zu sortieren und zu klassifizieren, die sie für beide Länder so gut wie möglich kennen. Studenten geben ein paar Tipps. Dieses Buch soll nicht allumfassend sein, aber es ist genau, detailliert und interessant.


\par
\noindent
Im letzten Semester förderten unsere Teammitglieder unter Leitung und Leitung von Projektsponsorin Frau Schneider die kulturelle Identität zwischen chinesischen und deutschen Studenten in unseren Hochschulen und Genossenschaftsinstitutionen und milderten die Kommunikationsbarrieren, die durch kulturelle Unterschiede verursacht wurden. Gleichzeitig wurden eine Reihe von Seminaren, Interviews, Fragebögen und Untersuchungen vor Ort zu den chinesisch-deutschen kulturellen Unterschieden durchgeführt, um sich an den internationalen Unterrichtsstil und das diversifizierte humanistische Umfeld anzupassen und zu integrieren. Das Projekt wurde schließlich beschlossen. Die Perspektive konzentriert sich auf sieben kulturelle Themen: Kleidung, Ernährung, Sport, Arbeit, Transport, Kultur, Gesundheit, webbasiertes, enzyklopädisches Schreiben, aus der Perspektive der Studenten, die eindeutig zeigen, dass wir exotisch sind Kulturelles Verständnis und Denken. Während des Projekts haben wir das Wissen des Projektmanagements voll genutzt und den Projektprozess durch Projektmanagement-Tools wie Projektauftrag, Zeitplan, PSP gesteuert, durch Absprache mit relevanten Materialien, Austausch mit deutschen Lehrern und Schülern und nach der tatsächlichen Lebenserfahrung bestimmt. Die spezifischen Inhalte jedes Themas und zeigen regelmäßig die Arbeitsergebnisse für Klassenkameraden und Lehrer. Am Ende des Projekts sind unsere Ergebnisse erfreulich: Unsere chinesisch-deutsche Enzyklopädie für kulturelle Unterschiede befindet sich auf der offiziellen Website des Kollegs und zeigt den chinesischen und deutschen Lehrern und Schülern einen Einblick in den chinesisch-deutschen Kulturgarten. Wir hoffen, dass durch die gemeinsamen Anstrengungen dieses Semesters die kulturelle Brücke zwischen China und Deutschland stärker und schöner wird.。
 \par 
 \noindent
 China ist ein Drache im Osten, hat eine lange Geschichte und eine tiefe Geschichte, Deutschland ist eine glänzende Perle in Europa und ein aufgehender Stern. Auf dem Weg der kontinuierlichen Entwicklung trafen sich die beiden Länder unerwartet und kannten sich. Die beiden Länder haben also eine Mischung von Kulturen: Die beiden Völker haben ein neugieriges Auge, um zu verstehen und tiefer zu gehen: Obwohl es zwischen der chinesischen und der deutschen Kultur viele Unterschiede gibt, haben beide Länder gegenseitiges Verständnis, gegenseitige Toleranz und gegenseitige Lerneinstellung. Die Bekanntschaft der Kultur macht die beiden Länder als Bruder eng miteinander verbunden. Vertrauen, es wird Zusammenarbeit, Zusammenarbeit und Win-Win geben. Vertrauen, basierend auf Verständnis, nur ein kulturelles Verständnis, vertieft das Verständnis beider Seiten und hat eine Grundlage, die die beiden Länder auf wissenschaftlicher, wirtschaftlicher und politischer Ebene vertrauenswürdiger macht. Kooperation und Win-Win sind ebenfalls selbstverständlich. Aus der Perspektive des großen Ganzen haben der kulturelle Austausch, die wechselseitigen Länder und die Verringerung der Unterschiede bei Wissen und Wissen zwischen den beiden Ländern die Entwicklung der beiden Länder wesentlich gefördert. Daher können sich China und Deutschland durch die Studie dieses Projekts besser verstehen, nach Gemeinsamkeiten suchen, dabei Unterschiede aufheben und gegenseitigen Nutzen erzielen.
\vspace{\baselineskip}
\begin{flushright}\noindent
HangZhou, Mai.  2019\hfill {\texttt Alle Mitgliede des Projekt4}\\
\end{flushright}

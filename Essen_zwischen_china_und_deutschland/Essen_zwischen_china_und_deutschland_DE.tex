\chapter{Essen zwisch China und Deutschland}
\startcontents[chapters]
%\printmyminitoc
\color{blue}
\hspace{4cm}\printcontents[chapters]{}{1}{\contentsmargin{6em}}
\noindent\hspace{-2cm}
\begin{tikzpicture}
\node[rectangle,rounded corners,align=left,fill=halfgray]{%	
	\begin{minipage}{0.7\linewidth}%minipage trick
	\color{white}%
	\printcontents[chapters]{}{1}{\contentsmargin{6em}}
	\end{minipage}};
\end{tikzpicture}
%\noindent\hspace{2cm}
\cleardoubleemptypage
Die Esskultur spielt eine zentrale Rolle in der Kommunikation mit dem chinesischen Volk: Durch die Analyse und das Studium der chinesischen und deutschen Esskultur können wir das kommunikative Verhalten mit dem deutschen Volk vorhersehen und die Probleme der Kommunikation lösen.
\section{Tee und Kaffe}
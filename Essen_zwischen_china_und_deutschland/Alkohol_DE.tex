\section{Weinkultur am Tisch}
Die chinesische Weinkultur ist eine einzigartige Kultur in China, deren Geschichte fast so lang ist wie die Geschichte der Menschheit: Bevor die chinesischen Schriftzeichen reiften, hatten die Chinesen die Technologie der Weinherstellung beherrscht. Es gibt auch viele Bücher, die dem Wein gewidmet sind, wie der ''Jiu Gao'' \footnote{"Jiu Gao" der westlichen Zhou-Dynastie. der westlichen Zhou-Dynastie 1046 v. Chr. B - 771 v. Chr. "Jiu Hao": Der Autor ist Zhou Gongdan. Es ist Chinas erste Alkoholverbot, dass die Menschen aufgefordert werden, nicht oft Alkohol zu trinken.} der westlichen Zhou-Dynastie, der ''Jiu Fu''und ''Jiu Zhen'' der Western Han Dynasty,  der ''Jiu Jie'' \footnote{Der Autor ist Ge Hong. Empfehlen Sie den Leuten, weniger zu trinken.} der östlichen Jin-Dynastie \footnote{317 n. Chr. - 420 n. Chr} und der ''Jiu Pu'' \footnote{Der Autor ist Dou Ping. Es erzählt die interessante Geschichte über Wein. Gleichzeitig werden viele Weinrezepte wie Cocktails aufgenommen.} der nördliche Song-Dynastie \footnote{ war von 960 bis 1279 die herrschende Dynastie im Kaiserreich China. Sie teilt sich in die „Nördliche“ und „Südliche“ Song-Dynastie. Die „Nördliche“ regierte von 960–1126 in Kaifeng, die „Südliche“  von 1126–1279 in Hangzhou.} usw. Man erkennt, dass Wein schon sehr früh zu einem wichtigen Element der chinesischen Kultur geworden ist: Die Weinkultur ist tief im Blut der Chinesen erwurzelt und hat weitreichenden Einfluss. In China kann man überreden ,ander leute zu trinken, um leute zu  respektieren und bestrafen. In Bezug auf Unterhaltung überreden sich die beiden Seiten der Transaktion rmalerweise zum Trinken,was für beide Seiten eine Versuchung ist, und in der Führung des Trinkens ist dies auch die Verkörperung dieser entalität. Überrede den Trinker, über deinen Gehorsam zu urteilen, indem du beobachtest, ob du die Anweisungen befolgst, die er wünscht, dass du weiter trinkst und ob du dich zwingen kannst, für die ''Szene'' zu trinken. In vielen Büchern gibt es Aufzeichnungen über Wein- und Trinkkultur, im “ShiJing ”\footnote{es sagt auch Buch der Lieder und ist ein  berühmter Buch in China. Es ist die älteste Sammlung von chinesischen Gedichten und die größte aus vorchristlicher Zeit. Konfuzius soll, der Tradition nach, die Lieder aus einem Fundus von 3000 Gedichten ausgewählt und in ihren jetzigen Zustand gebracht haben, dies ist jedoch eher eine Legende als eine Tatsache. Entstanden ist das Shijing zwischen dem 10. und dem 7. Jahrhundert v. Chr.} finden sich mehr als 20 Referenzen, und der Wein ist mit rituellen, sozialen, Freizeit- und anderen Bedeutungen ausgestattet, die die spezifische patriarchalische Ordnung und die menschlichen Beziehungen widerspiegeln.Die Überzeugungskultur in Chinas politischen und gesch?ftlichen Kreisen muss zwei Ziele erreichen: Compliance-Tests und Tests in Treu und Glauben. Es erfordert einen Preis, um eine Beziehung aufrecht zu erhalten. Betrunken ist der Preis. Der hässliche Zustand nach Betrunkenheit ist eine kleine Dosis von Sicherheiten, die zwischen den Menschen nicht voll vertrauenswürdig ist, aber wenn eine kooperative Beziehung hergestellt werden muss. Die Deutschen trinken sehr  geschmackvoll und die Weine werden im Allgemeinen in drei Kategorien unterteilt: Aperitifs, Tafelweine und Trank nach Essen. Die Deutschen achten beim Essen auf die Kombination von Speisen und Wein. Wenn Sie ein Fischsteak essen, trinken Sie auf jeden Fall Wei\ss wein, wenn Sie ein Beefsteak essen, trinken Sie auf jeden Fall Rotwein. Deutsche im  deutsche Tisch war ruhiger und zu trinken, aber es gab keinen Lärm. Um den Enthusiasmus auszudrücken, überzeugen die Chinesen oft den Wein, aber dieses Phänomen ist auf dem deutschen Weintisch nicht sichtbar. Die Chinesen gaben an,der Wein zu trinken um Ziel erreichen. Die Deutschen hätten gewöhnlich Freunde zu machen am Weintisch.


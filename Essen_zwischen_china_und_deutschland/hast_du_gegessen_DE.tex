\section{Hast du gegessen}
Grundsätzlich treffen sich in allen Ländern im Allgemeinen zwei Menschen,  die beiden treffen sich und müssen fragen: ,,Wie geht' s''. Antwort: ,,Dank,  gut und dir. '' Das Gespräch zwischen den beiden ist wie ein Leitfaden für das Schreiben von Nachrichten.  Dann werden wir über das Geschäft sprechen.  Es ist auch in Deutschland so, und wenn wir uns in China treffen, besonders wenn sich Bekannte treffen, lautet der erste Satz ,,Essen Sie? '' Dies ist eine sehr aufrichtige Begrü\ss ung an China, nicht nur Höflichkeit.
\mypar
Wenn die Begrü\ss ung für die Chinesen psychisch anstrengend ist, ist es für die andere Partei sehr peinlich, sich gegenseitig zu beantworten, weshalb der Ausdruck : ,,essen Sie? '' Seine klassischen Merkmale fortsetzt. Die öffentliche Begeisterung der Öffentlichkeit in den Augen der Öffentlichkeit spiegelt nicht nur ihre Begeisterung für andere wider, sondern macht die andere Partei nicht peinlich, sondern schlie\ss t auch die Beziehung zur anderen Seite.
\mypar
Diese Begrü\ss ung spiegelt die Bedeutung der chinesischen Ernährung für die Chinesen wider. Die sogenannten Menschen essen für den Himmel, denn das Essen der Chinesen ist ein gro\ss es Ereignis, besonders in Zeiten von Versorgungsengpässen. Die Esskultur nimmt in der Kommunikation mit den Chinesen eine zentrale Stellung ein: Das Bankett des Gastgebers gibt den Gästen oft einen Toast und ein Gericht, um die Herzliche Freundschaft  zu zeigen. In China ist die Ernährung auf etwas angewachsen, das fast alle anderen Formen besonders geistlichen Formen bekommen.
\mypar
Die traditionellen chinesischen Kochmethoden unterscheiden sich je nach den Hauptküchen. Dasselbe Gericht hat unterschiedliche Geschmacksrichtungen aufgrund der Expertise und Vorlieben des Küchenchefs und sogar aufgrund der wechselnden Stimmung des Küchenchefs. Dies spiegelt auch die chinesische Kultur wider, passt sich der Natur an, schätzt die Erfahrung und verachtet die Eigenschaften innovativer Experimente.
\mypar
,,Essen'' ist in Deutschland nur ein notwendiges Kommunikationsmittel und ein Kommunikationsmittel. Herr Lin Yutang \footnote{ein Berühmter chinesischer Schriftsteller und Übersetzer} sagte einmal. In der westlichen Welt gibt es andere Ernährungskonzepte als in China: Die Briten und Amerikaner verwenden nur ,,eat'', um Kraftstoff in eine biologische Maschine zu injizieren, um einen normalen Betrieb zu gewährleisten. Solange sie essen, können sie einen gesunden und starken Körper erhalten, der ausreicht, um Bakterien und Krankheiten vorzubeugen. es git keine andere wichtige Funktion.  Aus kultureller Sicht kann man sehen, dass es in westlichen Ländern nur eine einfache kommunikative Rolle ist und nicht wie in China mehr und wichtigere Missionen erhalten wurden.

\subsection{Theme-Erweiterung}
Die chinesische Alltagssprache hat nicht nur Grü\ss e, sondern hat auch einen gew\"ohnlichen Ausdruck:
\begin{enumerate}
	\item Beruf :  Reisschale
	\item Glück : Trunken
	\item neiden:  Essigessenz trinken
\end{enumerate}
\section{AA制和抢单}
AA制源于英文“go dutch”,意思是个人平均分摊所需费用,通常用于饮食聚会及旅游等共同消费的场合,人们在消费后分摊消费总额,互不相欠。普遍的说法是,AA制起源于荷兰,在16-17世纪荷兰和威尼斯的海上贸易十分发达,意大利人与荷兰人在各个地方来回奔波,流动性很大,为了避免各自吃亏,在他们中间就产生了聚时交流信息,散时各付资费的传统。
而抢单,顾名思义,意为抢着买单,与AA制不同,通常是在吃晚饭结账时,一桌人抢着买单。更有甚者,为抢占先机,锅里饭菜还没吃完,就找个上个厕所的借口,偷偷到柜台处结账买单。
在西方社会,在亲朋好友之间的聚会中,人们实行AA制是非常普遍的。但是在中国,AA制一般只被青年人,特别是大学生和白领所接受,传统的中国中老年人相较于AA制,更习惯于在聚会时替人买单。
究其原因,一方面西方人的价值观更偏向于利益至上。上至国家,下至个人,利益至上是西方社会普遍认同的价值观。因此,西方人上餐厅吃饭,就算两人感情再好,吃完了各付个的,这非常的自然,没有人会认为这不妥。只有这样,才能体现对别人和对自己私有财产的尊重,才能保全各自在金钱上的利益。而与之相反的是,尽管中国人普遍接受了派对、各种西方节日等丰富的西方生活方式,但AA制却始终未能流行开来,原因之一就是AA制与中国人看重人情、面子的文化传统相违背。看重情义、面子是中国人的性格特点。这种特点体现在对父母,对兄弟,师长,朋友等方方面面中。对中国人来说,很难接受与朋友出去吃饭,还要各付各的。这样子做会显得自己吝啬,也会被周遭人侧目相看。另外,中国人好客,爱请客。如果人们热热闹闹地围着一张大桌子,欢谈畅饮,最后结账时却分摊消费,这对中国人来说是丢面子的事,而面子,正是中国人最看重的东西。所以,许多国人宁愿打肿脸充胖子,也要抢着付钱,积极程度可与打架相当。
\par
另一方面,在中国,吃饭有着独特的文化内涵。中国人喜欢把事情搬到饭桌上解决。求人办事时会请人吃饭,答谢他人时会请人吃饭,送行时也会请人吃饭,如此这般。关于中国人请客吃饭时抢单的原因,一方面是因为纯粹的感情,另一方面也是因为人们为了办事而不得不进行应酬。请客在中国实际上是一种长期的投资,除了投入金钱外,也投入了长期的人情。这与西方公平保证双方的利益的AA制有着本质的区别。一般来说,被请客之人往往会有一种吃人嘴软的感觉,觉得欠了别人人情,以后被求帮忙也就不好意思拒绝了。人情是中国人实现利益最重要的手段之一,在中国,即使人人都明白这个道理,也却一次次掉入这个陷阱中,让请客之人百试不爽。而相对的,西方人不喜欢欠他人人情,事情必须一件件的算清楚,他们不觉得AA制让人尴尬,若你有事求他,而他不能从中得到好处,他可以很干脆地回绝你,因为在他们眼中,只有利益是最重要的,人情,面子,都是虚的。
而抢单文化在中国也并非是一伙人吃饭每次到了结账关头次次抢单,通常是这一次我请大家吃一顿,下一次你回请我吃一顿。这样子既有了面子,也有了利益的保障。在实际生活中,可能存在着有些朋友,他们家庭条件不好,但是感情到位了,大家就一直不让他买单,这也是普遍的事,真正的朋友之间,谁也不会在乎这点饭钱,抢单文化在这里也不再是对于人情的争抢,而是朋友之间的关怀与暖意。
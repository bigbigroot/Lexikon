\section{zusammen und getrennt essen}
\mypar
Es gibt viele kulturelle Unterschiede zwischen China und Deutschland, die sowohl durch geographische Faktoren als auch durch die Entwicklung der Menschheitsgeschichte bestimmt werden. Beginnen wir mit der Perspektive der Esskultur, suchen wir die kulturellen Unterschiede zwischen den beiden Ländern und erleben wir die exotischen Bräuche Chinas und Deutschlands. Wenn es um die Ernährung geht, möchte ich gerne die Ess-Etikette in China und Deutschland erwähnen, der wichtigste Unterschied ist das chinesische Essenssystem und das deutsche Essenssystem. Lassen Wir uns mit dem Essen und dem Essen beginnen und den kulturellen Charme zwischen China und Deutschland verstehen.
\mypar
 Fünftausend Jahre chinesische Geschichte, in der turbulenten Flusses der Geschichte, die nicht nur von der Ablösung der Dynastie begleitet wurde, sondern auch von Veränderungen im Lebenskonzept der Menschen. Es gibt ein Sprichwort, dass "die Leute Nahrung für den Himmel nehmen". Essen in China ist eine große Sache, die Chinesen haben es nie gewagt, zu vernachlässigen. Während der Zhou-, Qin- und Jin-Dynastien achteten die Leute auf die Etikette der Mahlzeit, saßen auf den Sitzen von zehn Personen und warteten darauf, dass der Koch das Essen brachte, das an den jeweiligen kleinen Tischen geliefert wurde. Egal wie hungrig es während dieser Zeit sein sollte, jeder sollte in Gefahr sitzen und ernst bleiben. Nach den "Historical Records of Meng Tseng-Ji", einem Gast, der die Regeln nicht versteht, weil das Essen auf seinem Tisch nicht lecker genug war , verlor er während des Essens die Geduld an den Besitzer, aber er sah den edlen Besitzer, seines Essen an seinem Tisch wie eigenes war, schämte er sich. Dies ist auch das lebendigste Foto des Essensystems in der chinesischen Geschichte, eine Person besitzt einen Tisch, die einen bestimmten Abstand zueinander einhalten. Die Zeit vergeht, nach Eintritt in die Tang- und Song-Dynastien, mit der Verbesserung der sozialen Produktivität, der Wechsel des Essensystems wurde möglich. Tang-Dynastie unterschied sich von allen Dynastien der Geschichte,die Li Tang Zivilisation gegründete, Es ist ein Symbol für Wohlstand , Harmonie und Toleranz. Die ethnischen Minderheiten passierten den Hexi-Korridor und durchquerten den Dunhuang und Shanhai-Zoll nach Changan, der internationalen Metropole der Zeit. Sie bringen nicht nur exotische Produkte mit, die die Menschen noch nie zuvor gesehen haben, sondern sie bringen auch ihre Diät und ihre Bräuche mit. Sie kennten nicht, dass sie bald die chinesische Nation mit ihrer alten Geschichte verändern und auf ihrem Tisch eine weitreichende Revolution der Ernährungskultur auslösen werden. Die Hu-Leute, die aßen und tranken skrupellos auf den großen Tischen und Stühlen, erweckten die tiefe Seele des chinesischen Volkes und den starken Wunsch, vereint und von den Großen getrennt zu werden. Menschen verlieben sich unwillkürlich in diese hohen Tische und Stühle. Seitdem ist der hohe Tisch und Stuhlhocker, um die Familien zum Essen kann, zum Mainstream des gesellschaftlichen Lebens der Menschen geworden. Vom alten Essenssystem bis zum aktuellen Essenssystem ist es eine Transformation der chinesischen Zivilisation und auch eine Manifestation der chinesischen Kultur: (1) Die innige Beziehung um Tisch sitzen ist Ausdruck des gegenseitigen Respekts und des gegenseitigen Respekts und der traditionellen Tugenden der Fürsorge für andere sowie eine Repräsentation der Vereinigung der Vereinigung, die zeigt, dass die Menschen die Kohäsion der Nation und die Harmonie der Gesellschaft erwarten.  (2) Menschen kommunizieren während des Essens miteinander und verstehen sich. (3) Das Essenssystem in China bietet eine große Auswahl an Speisen, aus denen die Menschen wählen können, und die Menschen haben mehr Auswahl an Gerichten. 

\mypar
Das Essen-Sharing-System hat sich bis heute im Westen fortgesetzt. Die Literatur dokumentiert von der Renaissance bis zur viktorianischen Ära, wo die europäischen Aristokraten das Geschirr an den Tisch brachten und die großen Teller mit Essen auf den Teller jedes Menschen verteilten. Die Zivilisten verteilten das Essen in der Küche und brachten es dann zu jedem. Der Grund, warum das Essenssystem zwischen Adel und Zivilisten so beliebt sein kann. Hauptsächlich, weil (1)  Menschen auf die Lebensmittelhygiene achten (2) Der einst durch Europa verwüstete schwarze Tod hat die Leben unzähliger Menschen gefangen genommen und diese Angst vor dem schwarzen Tod ist tief in den Herzen der Menschen verwurzelt. (3) Die erste und die zweite industrielle Revolution haben die Produktivität der Gesellschaft stark gefördert und die Menschen reicher gemacht, wodurch die Esskultur der Reichen und Adligen unter den Bürgern beliebt wurde. Ein solches System zum Teilen von Mahlzeiten spiegelt auch ein Merkmal der europäischen Kultur wider, dh die Menschen interferieren nicht miteinander und tragen ihre Verantwortung.

\mypar
Essenssystem ist ein großer kultureller Unterschied zwischen China und Deutschland. Deutschland: Fokus auf individualistische Kultur, Unabhängigkeit, Respekt für die Privatsphäre, rationale Analyse der Vor- und Nachteile. China: Chinesen konzentrieren sich auf Kollektivismus, sie respektieren sich und kümmern sich um andere. Chinesen geben gerne den Kunden Gericht. Chinesen glauben, dass dies ein Ausdruck des Respekts für Menschen ist. Während Chinesen, die an einem großen Tisch essen, zwangsläufig aufstehen und Gemüse aufheben müssen. Stattdessen denken sie, dass dies eine Manifestation von Intimität ist. Für Deutschen haben sie das Gefühl, dass der private Raum verletzt wurde.
Aus den Unterschieden in der Tischkultur zwischen China und Deutschland können wir die kulturellen Merkmale der beiden Völker erkennen. Egal ob China oder Deutschland, es gibt keine bessere oder schlimmere Kultur, diese Kultur sind die Kristallisation der Weisheit der Menschen in China und Deutschland. Sie sind wie Sterne am Nachthimmel, übersät mit der riesigen Galaxie der menschlichen Zivilisation.
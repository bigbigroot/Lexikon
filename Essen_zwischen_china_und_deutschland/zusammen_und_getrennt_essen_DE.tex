\section{zusammen und getrennt essen}
Es gibt viele kulturelle Unterschiede zwischen China und Deutschland, die sowohl durch geographische Faktoren als auch durch die Entwicklung der Menschheitsgeschichte bestimmt werden. Beginnen wir mit der Perspektive der Esskultur, erkunden Sie die kulturellen Unterschiede zwischen den beiden Ländern und erleben Sie die exotischen Bräuche Chinas und Deutschlands. Wenn es um die Ernährung geht, muss ich die Ess-Etikette in China und Deutschland erwähnen. Das chinesische Mahlsystem und das deutsche Mahlsystem. Lassen Sie uns den ganzen Leoparden sehen, beginnen Sie mit dem Essen und dem Essen und verstehen Sie den kulturellen Charme zwischen China und Deutschland.

\par
Fünftausend Jahre chinesische Geschichte, in der turbulenten Geschichte des Flusses, die nicht nur von der Ablösung der Dynastie begleitet wurde, sondern auch von Veränderungen im Lebenskonzept der Menschen. Die Menschen essen für den Tag, in China zu essen ist eine große Sache, die Chinesen haben es nie gewagt, zu vernachlässigen. Während der Zhou-, Qin- und Jin-Dynastien achteten die Leute auf die Etikette der Mahlzeit, saßen auf den Sitzen von zehn Personen und warteten darauf, dass der Koch das Essen brachte, das an den jeweiligen kleinen Tischen an den Tisch geliefert wurde. Egal wie hungrig es während dieser Zeit sein sollte, jeder sollte in Gefahr sitzen, ernst bleiben und den Mund des Mundes zurück in den Magen schlucken. Nach den "Historical Records Meng Mengjun Biography", einem Gast, der die Regeln nicht versteht, weil das Essen auf seinem Tisch nicht lecker genug ist, verlor er während des Essens die Geduld an den Besitzer, aber er sah den edlen Besitzer, das Essen an seinem Tisch und Wenn du das Gleiche isst, schämst du dich. Dies ist auch das lebendigste Foto des Essensystems in der chinesischen Geschichte, eine Person und ein Tisch, die einen bestimmten Abstand zueinander einhalten. Die Zeit vergeht, nach Eintritt in die Tang- und Song-Dynastien, mit der Verbesserung der sozialen Produktivität, der Wechsel des Essensystems wurde möglich. In der Tang-Dynastie unterschied sich dieses von Li Tangming gegründete großartige Reich von allen Dynastien der Geschichte: Es ist ein Symbol für Wohlstand, ein Symbol für Wohlstand und ein Symbol für Harmonie und Toleranz. Ein paar berühmte Leute kamen durch den Hexi-Korridor und durch den Zoll Dunhuang Jinshan nach Chang'an, der damaligen internationalen Metropole. Sie bringen nicht nur exotische Produkte mit, die die Menschen noch nie zuvor gesehen haben, sondern sie bringen auch ihre Diät und ihre Bräuche mit. Sie kennen ihre Stühle nicht, und sie werden bald die chinesische Nation mit ihrer alten Geschichte verändern und auf ihrem Tisch eine weitreichende Revolution der Ernährungskultur auslösen. Nachdem sie es gesehen hatten, standen die Hu-Leute, die in diesen Reihen verteilt wurden, auf dem großen Tisch und hatten nach dem Essen und Trinken keine Skrupel. Sie erweckt die Tiefen der Herzen des chinesischen Volkes, von der Teilung bis zur Einheit, vom Unterschied zur Einheit des starken Verlangens. Menschen verlieben sich unwillkürlich in diese hohen Tische und Stühle. Seitdem ist der hohe Tisch und Stuhlhocker, die Familie rund um den Tisch und das Essen, zum Mainstream des gesellschaftlichen Lebens der Menschen geworden. Von der Geschichte des Essenssystems bis zum aktuellen Essenssystem ist es eine Transformation der chinesischen Zivilisation und auch eine Manifestation der chinesischen Kultur: (1) Der intime Beziehungsrunde Tisch und das Round and Round sind ein Gentleman wie eine Jade, gegenseitige Achtung und gegenseitige Anpassung Für andere sorgen, die Tugenden der Gegenwart demütigen. Es ist auch eine symbolische Darstellung der Großen Einheit, die zeigt, dass die Menschen erwarten, den Zusammenhalt der Nation und die Harmonie der Gesellschaft zu steigern. (2) Zusammenarbeit fördern Wenn Menschen eine Mahlzeit teilen, fördern sie die Kommunikation, das gegenseitige Verständnis und die Zusammenarbeit. (3) Streitigkeiten lösen Widersprüche lösen und Harmonie steigern. (4) Auswahl der Vielfalt Die Vielfalt der Lebensmittel ist für die Menschen die Wahl: Manche mögen es, manche mögen es nicht und sie schließen sich nicht aus und tolerieren einander. (5) Es ist teuer und alles ist gut: Während der Ferien verhandelt jeder über die Auswahl der Speisen, respektiert sich gegenseitig und ist tolerant.
\par
Das Essen-Sharing-System hat sich bis heute im Westen fortgesetzt. Die Literatur dokumentiert von der Renaissance bis zur viktorianischen Ära, wo die europäischen Aristokraten das Geschirr an den Tisch brachten und die großen Teller mit Essen auf den Teller jedes Menschen verteilten. Und den Zivilisten in der Küche wird das Essen verteilt und dann an alle. Der Grund, warum das Essenssystem zwischen Adel und Zivilisten so beliebt sein kann. Hauptsächlich, weil (1) die Menschen auf die Lebensmittelhygiene achten (2) Der einst durch Europa verwüstete schwarze Tod hat das Leben unzähliger Menschen gefangen genommen und diese Angst vor dem schwarzen Tod ist tief in den Herzen der Menschen verwurzelt. (3) Mit dem Aufkommen der ersten und zweiten industriellen Revolution wurde die Produktivität der Gesellschaft stark gesteigert, und die Menschen wurden reicher, wodurch die Esskultur der Reichen und des Adels unter den Bürgern populär wurde. Ein solches System zum Teilen von Mahlzeiten spiegelt auch ein Merkmal der europäischen Kultur wider: Die Menschen stören sich nicht und jeder trägt seine Verantwortung.
\par
Essen und Abendessen sind ein gro\-ss er kultureller Unterschied zwischen China und Deutschland. Deutschland: Fokus auf individualistische Kultur, Unabhängigkeit, Respekt für die Privatsphäre, rationale Analyse und das Für und Wider anderer. Die chinesische Seite: Achten Sie auf Kollektivismus, respektieren Sie gegenseitigen Respekt, lieben Sie und unterstützen Sie die Fürsorge für andere. Chinesen fügen ihren Gästen gerne Essen hinzu, und die Chinesen glauben, dass dies die Verkörperung von Respekt für Menschen und Menschen ist. Und wenn es in Deutschland ist, befürchte ich, dass die Freiheit und Autonomie des Einzelnen verletzt wird. Gleichzeitig ist es unvermeidlich, dass Chinesen, wenn sie dasselbe Gericht auf einem großen Tisch essen, mehr und mehr ernähren. Dies ist eine Manifestation von Harmonie und Toleranz in der Kultur. Für die Deutschen haben sie das Gefühl, dass der private Raum verletzt wurde. Weil die deutsche Kultur der Privatsphäre mehr Aufmerksamkeit schenkt.
Aus den Unterschieden in der Tischkultur zwischen China und Deutschland können wir die kulturellen Merkmale der beiden Völker erkennen. Egal ob China oder Deutschland, es gibt keine gute oder schlechte Kultur, diese Kulturen sind die Kristallisation der Weisheit der Menschen in China und Deutschland. Sie sind wie Sterne am Nachthimmel, übersät mit der riesigen Galaxie der menschlichen Zivilisation.
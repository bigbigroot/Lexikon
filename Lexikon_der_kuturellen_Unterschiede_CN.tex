\documentclass[a4paper, utf8]{book}
\usepackage{xeCJK}
\usepackage{amsmath}
\usepackage{tikz}
\usepackage{epigraph}
\usepackage{lipsum}
\usetikzlibrary{trees}
\usepackage{listings}

\usepackage{graphicx}
\usepackage{float}
\usepackage{tocloft}

\renewcommand\epigraphflush{flushright}
\renewcommand\epigraphsize{\normalsize}
\setlength\epigraphwidth{0.7\textwidth}

\definecolor{titlepagecolor}{cmyk}{1,.60,0,.40}

\DeclareFixedFont{\titlefont}{T1}{ppl}{b}{it}{0.5in}

\makeatletter                       
\def\printauthor{%                  
	{\Huge  \@author}}              
\makeatother
\author{%
	Projeckt 4 \\
	CDAI \\
	\texttt{}
}

% The following code is borrowed from: https://tex.stackexchange.com/a/86310/10898

\newcommand\titlepagedecoration{%
	\begin{tikzpicture}[remember picture,overlay,shorten >= -10pt]
	
	\coordinate (aux1) at ([yshift=-15pt]current page.north east);
	\coordinate (aux2) at ([yshift=-410pt]current page.north east);
	\coordinate (aux3) at ([xshift=-4.5cm]current page.north east);
	\coordinate (aux4) at ([yshift=-150pt]current page.north east);
	
	\begin{scope}[titlepagecolor!40,line width=12pt,rounded corners=12pt]
	\draw
	(aux1) -- coordinate (a)
	++(225:5) --
	++(-45:5.1) coordinate (b);
	\draw[shorten <= -10pt]
	(aux3) --
	(a) --
	(aux1);
	\draw[opacity=0.6,titlepagecolor,shorten <= -10pt]
	(b) --
	++(225:2.2) --
	++(-45:2.2);
	\end{scope}
	\draw[titlepagecolor,line width=8pt,rounded corners=8pt,shorten <= -10pt]
	(aux4) --
	++(225:0.8) --
	++(-45:0.8);
	\begin{scope}[titlepagecolor!70,line width=6pt,rounded corners=8pt]
	\draw[shorten <= -10pt]
	(aux2) --
	++(225:3) coordinate[pos=0.45] (c) --
	++(-45:3.1);
	\draw
	(aux2) --
	(c) --
	++(135:2.5) --
	++(45:2.5) --
	++(-45:2.5) coordinate[pos=0.3] (d);   
	\draw 
	(d) -- +(45:1);
	\end{scope}
	\end{tikzpicture}%
}

\begin{document}
	\begin{titlepage}
		
		\noindent
		\titlefont Lexikon der kuturellen Unterschiede zwischen Deutschland und China\par
		\epigraph{Das "Lexikon der kuturellen Unterschiede" ist ein Projekt im Rahmen des Studiums am CDAI, das Studierende innerhalb der Lehrveranstaltung "Inter\-disziplinäres Projekt" durchgeführt haben. Dabei war die Aufgabe, eine Enzyklopädie aus freien Inhalten auf\-zubauen, die mit der Zeit wachsen darf. Du darfst gerne dazu beitragen!}%
		{\textit{Über Lexikon der kutrurellen Unterschiede}\\ \textsc{}}
		\null\vfill
		\vspace*{1cm}
		\noindent
		\hfill
		\begin{minipage}{0.35\linewidth}
				\center{\printauthor}
		\end{minipage}
		%
		%\begin{minipage}{0.02\linewidth}
			%\rule{1pt}{125pt}
		%\end{minipage}
		\titlepagedecoration
	\end{titlepage}
	\renewcommand\thefigure{\thesection-\arabic{figure}}
	\renewcommand\thetable{\thesection-\arabic{table}}
	\makeatletter
	\@addtoreset{table}{section}
	\makeatother
	\makeatletter
	\@addtoreset{figure}{section}
	\makeatother
	
\end{document}
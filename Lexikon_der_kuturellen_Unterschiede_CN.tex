\documentclass[headinclude,footinclude,openany]{scrbook}
\usepackage{xeCJK}
%\usepackage{amsmath}
%\usepackage{xcolor}
%\usepackage{titlesec}

\usepackage{tikz}
\usepackage{epigraph}
\usepackage{lipsum}
\usetikzlibrary{trees}
\usepackage{listings}
\usepackage{graphicx}        % standard LaTeX graphics tool
\usepackage{float}
\usepackage{tocloft}
%\usepackage{fancyhdr}
\usepackage[eulerchapternumbers]{classicthesis}
\usepackage{arsclassica}
\usepackage{changepage}
\strictpagecheck

\renewcommand\epigraphflush{flushright}
\renewcommand\epigraphsize{\normalsize}
\setlength\epigraphwidth{0.7\textwidth}

\definecolor{titlepagecolor}{cmyk}{1,.60,0,.40}

\DeclareFixedFont{\titlefont}{T1}{ppl}{b}{it}{0.5in}

\makeatletter                       
\def\printauthor{%                  
	{\@author}}              
\makeatother
\author{%
	\large \textbf{Chinese-German Institute for Applied Engineering(CDAI) } \\ 
	\textbf{}\\
	\large \textbf{webseit: }\texttt{http://cdai.zust.edu.cn/} 
}

\makeatletter
%\renewcommand\thepart{\arabic{part}}
\def\@part[#1]#2{%
	\ifnum \c@secnumdepth &gt;-2\relax
	\refstepcounter{part}%
	\addcontentsline{toc}{part}{\thepart\hspace{1em}#1}%
	\else
	\addcontentsline{toc}{part}{#1}%
	\fi
	\markboth{}{}%
	\reset@font
	\parindent \z@
	\vspace*{10\p@}%
	\hbox{%
		\vbox{%
			\hsize=7mm%
			\begin{tabular}{@{}p{7mm}@{}}
				\makebox[7mm]{\scshape\strut\small\partname}\\
				\makebox[7mm]{\cellcolor{black}\Huge\color{white}\bfseries\strut\thepart\rule[-4cm]{0pt}{4cm}}%
			\end{tabular}%
			\makebox(0,0){\put(-10,-100){\fbox{\phantom{\rule[-4cm]{7mm}{4cm}}}}}
		}%
		\kern-2pt
		\vbox to 0pt{%
			\tabular[t]{@{}p{1cm}p{\dimexpr\hsize-2.1cm}@{}}\hline
			& \Huge\itshape\rule{0pt}{1.5\ht\strutbox}#1\endtabular}%
	}%
	\cleardoublepage
	%  \vskip 100\p@
}
\makeatother
% The following code is borrowed from: https://tex.stackexchange.com/a/86310/10898

\newcommand\titlepagedecoration{%
	\begin{tikzpicture}[remember picture,overlay,shorten >= -10pt]
	
	\coordinate (aux1) at ([yshift=-15pt]current page.north east);
	\coordinate (aux2) at ([yshift=-410pt]current page.north east);
	\coordinate (aux3) at ([xshift=-4.5cm]current page.north east);
	\coordinate (aux4) at ([yshift=-150pt]current page.north east);
	
	\begin{scope}[titlepagecolor!40,line width=12pt,rounded corners=12pt]
	\draw
	(aux1) -- coordinate (a)
	++(225:5) --
	++(-45:5.1) coordinate (b);
	\draw[shorten <= -10pt]
	(aux3) --
	(a) --
	(aux1);
	\draw[opacity=0.6,titlepagecolor,shorten <= -10pt]
	(b) --
	++(225:2.2) --
	++(-45:2.2);
	\end{scope}
	\draw[titlepagecolor,line width=8pt,rounded corners=8pt,shorten <= -10pt]
	(aux4) --
	++(225:0.8) --
	++(-45:0.8);
	\begin{scope}[titlepagecolor!70,line width=6pt,rounded corners=8pt]
	\draw[shorten <= -10pt]
	(aux2) --
	++(225:3) coordinate[pos=0.45] (c) --
	++(-45:3.1);
	\draw
	(aux2) --
	(c) --
	++(135:2.5) --
	++(45:2.5) --
	++(-45:2.5) coordinate[pos=0.3] (d);   
	\draw 
	(d) -- +(45:1);
	\filldraw[red,line width=0,rounded corners=0] (0,-.5) rectangle (13,-.55);
	\end{scope}
	\end{tikzpicture}%
}

\definecolor{halfgray}{gray}{0.55}%定义所需颜色
\newcommand\anglei{-45}%定义角度
\newcommand\angleii{45}
\newcommand\angleiii{225}
\newcommand\angleiv{135}
%绘制版面镶边代码
\newcommand\chapterdecoration{%
	\begin{tikzpicture}[remember picture,overlay,shorten >= -10pt]
	\coordinate (aux1) at ([yshift=-15pt]current page.north east);
	\coordinate (aux2) at ([yshift=-410pt]current page.north east);
	\coordinate (aux3) at ([xshift=-4.5cm]current page.north east);
	\coordinate (aux4) at ([yshift=-150pt]current page.north east);
	\checkoddpage
	\ifoddpage
	\else
	\coordinate (aux1) at ([yshift=-15pt]current page.north west);
	\coordinate (aux2) at ([yshift=-410pt]current page.north west);
	\coordinate (aux3) at ([xshift=4.5cm]current page.north west);
	\coordinate (aux4) at ([yshift=-150pt]current page.north west);
	\renewcommand\anglei{-135}
	\renewcommand\angleii{135}
	\renewcommand\angleiii{-45}
	\renewcommand\angleiv{45}
	\fi
	\begin{scope}[halfgray!40,line width=12pt,rounded corners=12pt]
	\draw
	(aux1) -- coordinate (a)
	++(\angleiii:5) --
	++(\anglei:5.1) coordinate (b);
	\draw[shorten <= -10pt]
	(aux3) --
	(a) --
	(aux1);
	\draw[opacity=0.6,halfgray,shorten <= -10pt]
	(b) --
	++(\angleiii:2.2) --
	++(\anglei:2.2);
	\end{scope}
	\draw[halfgray,line width=8pt,rounded corners=8pt,shorten <= -10pt]
	(aux4) --
	++(\angleiii:0.8) --
	++(\anglei:0.8);
	\begin{scope}[halfgray!70,line width=6pt,rounded corners=8pt]
	\draw[shorten <= -10pt]
	(aux2) --
	++(\angleiii:3) coordinate[pos=0.45] (c) --
	++(\anglei:3.1);
	\draw
	(aux2) --
	(c) --
	++(\angleiv:2.5) --
	++(\angleii:2.5) --
	++(\anglei:2.5) coordinate[pos=0.3] (d);
	\draw
	(d) -- +(\angleii:1);
	\end{scope}
	\end{tikzpicture}%
}

\titleformat{\chapter}[block]%
{\normalfont\large\sffamily}%
{{\color{halfgray}\chapterNumber\thechapter%
		\hspace{10pt}\vline} }{10pt}%
{\spacedallcaps}[\chapterdecoration]

\begin{document}
	\begin{titlepage}
		
		\noindent
		\titlefont Lexikon der kuturellen Unterschiede zwischen Deutschland und China\par
		\epigraph{Heutzutage gewinnt der deutsch-chinesische Austausch immer mehr an Bedeutung, aber die kul\-turellen Unter\-schiede zwischen den beiden Seiten erschweren diese Kommunikation. Das Lexikon soll diese Kommunikation erleichtern und die Unterschiede klären.}%
		{\textit{Über Lexikon der kutrurellen Unterschiede}\\ \textsc{}}
		\null\vfill
		\vspace*{6cm}
		\noindent
		\hfill
		
		\begin{minipage}{\linewidth}
				\center{\printauthor}
		\end{minipage}
		%
		%\begin{minipage}{0.02\linewidth}
			%\rule{1pt}{125pt}
		%\end{minipage}
		
		\titlepagedecoration
		
	\end{titlepage}
	
	\phantom{s}
	\thispagestyle{empty}
	\cleardoublepage
	
	\begin{titlepage}
			\noindent
			\titlefont Lexikon der kuturellen Unterschiede zwischen Deutschland und China\par
			\epigraph{Heutzutage gewinnt der deutsch-chinesische Austausch immer mehr an Bedeutung, aber die kulturellen Unterschiede zwischen den beiden Seiten erschweren diese Kommunikation. Das Lexikon soll diese Kommunikation erleichtern und die Unterschiede klären.}%
			{\textit{Über Lexikon der kutrurellen Unterschiede}\\ \textsc{}}
			\null\vfill
			\vspace*{6cm}
			\noindent
			\hfill
			
			\begin{minipage}{\linewidth}
				\center{\printauthor}
			\end{minipage}
	\end{titlepage}
	
	\phantom{s}
	\thispagestyle{empty}
	\cleardoublepage
	\titlefont  Preface\par

\thispagestyle{empty}
 \vspace*{5cm}
 \normalfont
 中国和德国都是世界上具有重大影响力的国家,随着传播 通讯技术的改进,交通技术的进步和经济的高度全球化,两国的合作越来越频繁。 然而,由于文化背景的不同,两国人民在交 往的过程中不能够相互理解,,导致交际不能顺利进行。针对中德工程师学院和德国的应用技术大学之间日益频繁的交流活动,本项目成员作为中德工程师学院的学生希望将一些所闻所知的一些信息整理、归类,尽可能为两国的学生做一些提示。本书不求包罗万象,但求准确,详尽。
 
 
\vspace{\baselineskip}
\begin{flushright}\noindent
HangZhou, Mai.  2019\hfill {\texttt Alle Mitgliede des Projekt4}\\
\end{flushright}

	\tableofcontents
	\phantom{s}
	\thispagestyle{empty}
	\cleardoublepage
	\renewcommand\thefigure{\thesection-\arabic{figure}}
	\renewcommand\thetable{\thesection-\arabic{table}}
	\makeatletter
	\@addtoreset{table}{section}
	\makeatother
	\makeatletter
	\@addtoreset{figure}{section}
	\makeatother
	\chapter{Essen zwisch China und Deutschland}

\cleardoubleemptypage
饮食文化在与中国人民的交际中占有举足重轻的地位,通过对中德饮食文化的分析研究,,有利于帮助我们预见与德国人民的交际行为,并解决交际中所遇到的问题。

\end{document}